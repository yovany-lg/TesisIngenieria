			% P�gina sin encabezados ni pies de p�gina
\chapter*{Resumen}
\addcontentsline{toc}{chapter}{Resumen}
%\thispagestyle{empty}

Actualmente, el monitoreo remoto es ampliamente utilizado en la industria para monitorizar y controlar procesos. La monitorizaci�n y automatizaci�n remota se engloban en el concepto de SCADA. Los sistemas SCADA constan de elementos \emph{hardware} y \emph{software}, que permiten el acceso a datos remotos y el control de un proceso (en general industrial) mediante el uso de sistemas de comunicaciones.

Este proyecto de tesis tiene como objetivo dise�ar y construir un prototipo de sistema de monitoreo remoto de temperatura, pH y Ox�geno Disuelto para la planta de tratamiento de aguas residuales de la Universidad Tecnol�gica de la Mixteca.

El sistema propuesto consta de sensores de prop�sito industrial, almacena informaci�n de mediciones en un servidor de bases de datos y ofrece acceso v�a Web mediante una interfaz de usuario que permite visualizar el estado de los par�metros medidos de manera gr�fica y tabular.

Debido a la magnitud del sistema SCADA, para su desarrollo se emplea la metodolog�a de desarrollo para mejoramiento de procesos de producci�n y la metodolog�a de sistemas empotrados. Con ello se establecen las fases de desarrollo que producen un proyecto completo y modular.

Al final, se obtuvo un prototipo de sistema SCADA capaz de monitorizar una planta o proceso industrial usando diferentes interfaces de comunicaci�n e interfaces de usuario para la visualizaci�n de los registros de mediciones.

